%! TEX program = lualatex
\documentclass[a4paper]{article}

% Language
\usepackage{polyglossia}
\setmainlanguage{french}

% Maths
\usepackage{amsmath}
\usepackage{amssymb}
\usepackage{amsfonts}
\usepackage{esvect} % for \vv
\usepackage{tikz}

% Links
\usepackage{color}
\usepackage[pdfstartview=XYZ]{hyperref}
\hypersetup{
    breaklinks,
    colorlinks,
    citecolor=black,
    filecolor=black,
    linkcolor=black,
    urlcolor=blue
}

%%%%%%%%%%%%%%%%%%%%%%%%%%%%%%%%%%%%%%%%%%%%%%%%%%%%%%
%%%%%%%%%%%%%%%%%%%%%%%%%%%%%%%%%%%%%%%%%%%%%%%%%%%%%%
%%%%%%%%%%%%%%%%%%%%%%%%%%%%%%%%%%%%%%%%%%%%%%%%%%%%%%

%%%%%%%%%%%%%%%%%%%
%% Main document %%
%%%%%%%%%%%%%%%%%%%

\newtheorem{hypothesis}{Hypothèse}


\begin{document}
\tableofcontents
\newpage

\section{Introduction}
Ce document contient la documentation des calculs physiques liés au projet.

\section{Calculs de collision}
Dans toutes les sections suivantes, on s'intéresse aux collisions entre deux systèmes sur un intervalle de temps $[0, dt]$.

\subsection{Hypothèses}
En fonctions des cas, les hypothèses suivantes peuvent être utilisées pour simplifier:
\begin{hypothesis}
  \label{hyp:neglect_external_forces}
  La collision se produit sur un petit intervalle de temps $[t_0,t_0+dt]$ tel que les seules forces significatives
  sur cet intervalle sont les forces de collision, si celles-ci existent.
\end{hypothesis}

\begin{hypothesis}
  \label{hyp:rebound_heavy_body}
  Les collisions entre les deux corps sont de nature élastique avec un facteur de restitution $0 < k \leq 1$.
  De plus, l'un des deux corps a une masse bien plus élevée que l'autre.
\end{hypothesis}

\begin{hypothesis}
  \label{hyp:no_rotation}
  Les corps considérés ne peuvent pas subir de rotation.
\end{hypothesis}



\subsection{Rebond élastique}
On se place sous les hypothèses \ref{hyp:neglect_external_forces} et \ref{hyp:no_rotation}.

D'après \href{https://en.wikipedia.org/wiki/Elastic_collision}{Wikipedia},
le rebond élastique entre deux corps de masses respectives $m_1$ et $m_2$,
de vitesses respectives $\vv v_1$ et $\vv v_2$,
sur une surface de contact de normale $\vv n$ (orientée vers l'intérieur du premier corps) est:
\begin{equation}
  \begin{aligned}
    \vv v_1' = \vv{v_1} - \frac{2m_2}{m_1 + m_2} \left((\vv{v_1} - \vv{v_2}) \cdot \vv n \right) \vv n \\
    \vv v_2' = \vv{v_2} + \frac{2m_1}{m_1 + m_2} \left((\vv{v_2} - \vv{v_1}) \cdot \vv n \right) \vv n
  \end{aligned}
\end{equation}

Si on utilise de plus l'hypothèse \ref{hyp:rebound_heavy_body}, par exemple en supposant
$m_1$ petit devant $m_2$, on obtient simplement:
\begin{equation}
  \begin{aligned}
    \vv{v_1}' = \vv{v_1} - 2 \left((\vv{v_1} - \vv{v_2}) \cdot \vv n \right) \vv n \\
    \vv{v_2}' = \vv{v_2}
  \end{aligned}
\end{equation}

Si de plus le corps 2 est immobile, on obtient:
\begin{equation}
  \vv{v_1}' = \vv{v_1} - 2 \left(\vv{v_1} \cdot \vv n \right) \vv n \\
\end{equation}



\subsection{Disque mouvant et rectangle fixe}
On utilise les hypothèses \ref{hyp:neglect_external_forces},
\ref{hyp:rebound_heavy_body} et \ref{hyp:no_rotation}. Le rectangle est supposé de masse élevée.
\subsubsection{Convention de nommage}
\begin{center}
  \begin{tikzpicture}[scale=2.5]
    %% Building the box
    \coordinate (A) at (-1, 2);
    \coordinate (B) at (1, 2);
    \coordinate (C) at (1, 0);
    \coordinate (D) at (-1, 0);

    \draw (A) -- (B) -- (C) -- (D) -- (A);

    \draw (A) node[above left] {$A$} node {$\bullet$};
    \draw (B) node[above right] {$B$} node {$\bullet$};
    \draw (C) node[below right] {$C$} node {$\bullet$};
    \draw (D) node[below left] {$D$} node {$\bullet$};

    % Building the disc
    \coordinate (M) at (-2, 1);
    \coordinate (E) at (-2.5, 1);
    \coordinate (F) at (-2, 1.5);
    \coordinate (G) at (-1.5, 1);
    \coordinate (H) at (-2, 0.5);

    \draw (M) circle (0.5);
    \draw [dashed] (M) -- (F) node[midway,right] {$R$};
    \draw [->] (M) -- (-1.5, 0.5) node[midway, above right] {$\vv v$};

    \draw (M) node[below left] {$M$} node {$\bullet$};
    \draw (E) node[left] {$E$} node {$\bullet$};
    \draw (F) node[above] {$F$} node {$\bullet$};
    \draw (G) node[right] {$G$} node {$\bullet$};
    \draw (H) node[below] {$H$} node {$\bullet$};
  \end{tikzpicture}
\end{center}

\subsubsection{Rebond sur une arête}
En appelant $E'$ (resp. $F'$, $G'$, $H'$) la position de $E$ (resp. $F$, $G$, $H$) après $dt$,
on remarque que le disque sera en collision avec une arête ssi l'une de ces assertions est vérifiée:
\begin{enumerate}
  \item $[EE'] \cap [BC] \neq \emptyset$
  \item $[FF'] \cap [CD] \neq \emptyset$
  \item $[GG'] \cap [DA] \neq \emptyset$
  \item $[HH'] \cap [AB] \neq \emptyset$
\end{enumerate}

Soit $t_1$ (resp. $t_2$, $t_3$, $t_4$) la plus petite valeur telle que:
\[
  \left\{
  \begin{array}{l}
    E +t_1\vv v \in [BC] \\
    0 \leq t_1 \leq dt
  \end{array}
  \right.
\]
ou $t_1 = -1$ si une telle valeur n'existe pas (resp. équations similaires pour les autres arêtes).

L'arête de collision est celle dont le temps de collision est le plus faible.
On note $M_A$ le point d'intersection avec l'arête, et $\vv n_A$ la normale à cette arête.

\subsubsection{Collision sur un sommet}

On remarque que le disque peut également entrer en collision avec l'un des quatre sommets du rectangle.
Le rebond sera différent dans ce cas.

Soit $M_t$ la position du centre du cercle à l'instant $t$. On a $M_t = M + t \vv v$.

La collision d'un disque mouvant avec un point $P(x_P,y_P)$ a lieu s'il existe $t$ tel que:
\[
  PM_t^2 \leq R^2
\]
Soit en développant:
\begin{gather*}
    \vv{PM_t} \cdot \vv{PM_t} \leq R^2 \\
    (M + t\vv v - P)^2 \leq R^2 \\
    (\vv{PM} + t\vv v)^2 \leq R^2 \\
    PM^2 + t^2 v^2 + 2t\vv{PM}\cdot \vv v \leq R^2 \\
    t^2 v^2 + 2t\vv{PM}\cdot \vv v + PM^2 - R^2 \leq 0
\end{gather*}
On retrouve un trinôme du second degré dont le coefficient dominant est positif.

Si le coefficient dominant est nul (\textit{i.e.} disque immobile),
$t$ disparaît de l'équation:
\[
  PM^2 \leq R^2
\]

Supposons $v^2 > 0$.
L'existence de solution à cette inéquation est donc équivalent à l'existence de racines.
En notant $\Delta$ le discrimant du trinôme:
\[
  \Delta = 4\left( \vv{PM} \cdot \vv v \right)^2 - 4 (PM^2 - R^2)v^2
\]

En supposant $\Delta \geq 0$, les valeurs de $t$ donc comprises entre:
\[
  t_{\text{min},\text{max}} = - \vv{PM}\cdot \vv v \pm \frac{\sqrt{\Delta}}{2v^2}
\]

Les valeurs acceptables de $t$ sont celles de l'intervalle $[0,dt]$.
Le temps d'intersection est donc la plus petite de ces valeurs disponibles, soit:
\begin{itemize}
  \item Aucune si le trinôme n'a pas de solution, ou si $t_\text{min} > dt$ ou $t_\text{max} < 0$
  \item $\max(0, t_\text{min})$ sinon.
\end{itemize}

\subsubsection{Bilan}
Parmi toutes les intersections possibles calculées précédemment,
l'intersection réelle est celle dont le temps d'intersection est le plus faible.

Si deux intersections sont possibles (\textit{e.g.} $H + dt\vv v = A$),
on décide arbitrairement que le rebond sur le segment est prioritaire.

La vitesse est alors transformée par la relation:
\[
  \vv v' = \vv v - 2 \vv v \cdot \vv n \vv v = (1-2\vv v \cdot \vv n) \vv v
\]
où $\vv n$ est la normale à la surface de contact d'intersection, soit:
\begin{itemize}
  \item La normale à l'arête, orientée vers l'extérieur du rectangle,
        si la collision a lieu sur une arête
  \item La somme des normales des deux arêtes renormalisée,
        si la collision a lieu sur un sommet (l'intersection de deux arêtes)
\end{itemize}
Dans le cas d'un rectangle aligné sur les axes $(Ox)$ et $(Oy)$,
puisque les normales ont des coordonnées triviales ($\pm 1$ ou $0$),
le calcul de la vitesse \textit{a posteriori} est plus simple encore:
il suffit d'opposer l'abscisse ou l'ordonnée, voire les deux.


\subsection{Disque mouvant et demi-disque mouvant}
On utilise les hypothèses \ref{hyp:neglect_external_forces},
\ref{hyp:rebound_heavy_body} et \ref{hyp:no_rotation}. Le demi-disque est supposé de masse élevée.

\subsubsection{Convention de nommage}
\begin{center}
  \begin{tikzpicture}[scale=2.5]
    %% Building the half-disc
    \coordinate (C) at (0,0);
    \coordinate (G) at (0, 0.21); % 4*R/3pi, see below for computation

    \draw (0.5, 0) arc (0:180:0.5);
    \draw [dashed] (C) -- (0.35, 0.35) node[midway,right] {$R'$};
    \draw (-0.5, 0) -- (0.5, 0);

    \draw (C) node[below left] {$C$} node {$\bullet$};
    \draw (G) node[left] {$G$} node {$\bullet$};
    \draw [->] (G) -- (-1, 1) node[midway, above right] {$\vv v'$};

    % Building the disc
    \coordinate (M) at (-2, 1);

    \draw (M) circle (0.5);
    \draw [dashed] (M) -- (-2, 1.5) node[midway,right] {$R$};
    \draw [->] (M) -- (-1.5, 0.5) node[midway, above right] {$\vv v$};

    \draw (M) node[below left] {$M$} node {$\bullet$};
  \end{tikzpicture}
\end{center}

\subsubsection{Centre de gravité du demi-disque}

Commençons par localiser $G$. Supposons que $C$ soit l'origine du repère. Soit $H$ l'ensemble des points du demi-disque.
Alors:
\[
  \int_{M\in H} \vv{GM} \, dM = \vv 0.
\]
Soient $(x_G, y_G)$ les coordonnées de $G$. Cela revient à écrire:
\[
  \iint_{(x,y)\in H}
   \left(
     \begin{array}{c}
      x - x_G \\
      y - y_G
     \end{array}
   \right) \, dxdy = \vv 0.
\]
Soit en passant en coordonnées polaires:
\begin{gather*}
  \int_{r=0}^{r=R}
  r
  \int_{\theta=0}^{\theta=\pi}
   \left(
     \begin{array}{c}
      r\cos\theta - x_G \\
      r\sin\theta - y_G
     \end{array}
   \right) \, d\theta dr = \vv 0 \\
%
  \int_{r=0}^{r=R}
  r
   \left[
     \begin{array}{c}
      r\sin\theta - x_G\theta \\
      -r\cos\theta - y_G\theta
     \end{array}
   \right]_{\theta=0}^{\theta=\pi} \, dr = \vv 0 \\
%
  \int_{r=0}^{r=R}
  r
  \left(
    \begin{array}{c}
      - x_G\pi \\
      2r - y_G\pi
    \end{array}
 \right) \, dr = \vv 0 \\
%
  \left[
    \begin{array}{c}
      - x_G\pi\frac{r^2}{2} \\
      \frac{2r^3}{3} - y_G\pi\frac{r^2}{2}
    \end{array}
  \right]_{r=0}^{r=R} = \vv 0 \\
%
  \left(
    \begin{array}{c}
      - x_G\pi\frac{R^2}{2} \\
      \frac{2R^3}{3} - y_G\pi\frac{R^2}{2}
    \end{array}
  \right) = \vv 0 \\
\end{gather*}
Soit finalement:
\[
  \left\{
    \begin{array}{r c l}
      x_G &=& 0 \\
      y_G &=& \frac{4R}{3\pi} \approx 0.42 R
    \end{array}
  \right.
\]

\subsubsection{Collision sur la partie supérieure}
Intéressons-nous aux collisions possibles,
en supposant que seule la partie supérieure du demi-disque puisse entrer en collision.

On est ramené à une collision entre deux disques, respectivement de centre $M$ et de rayon $R$,
et de centre $C$ et de rayon $R'$.

Soit $M_t = M + t\vv v$ (resp. $M_t' = M' + t\vv v'$) la position de $M$
(resp. de $M'$) à l'instant $t$. Soit $r = \min(R,R')$. La collision a lieu à l'instant $t$ ssi:
\begin{gather*}
  M_tM_t' \leq r \\
  \left( \vv{M_t M_t'} \right)^2 \leq r^2 \\
  \left( M' + t\vv v' - M - t \vv v \right)^2 \leq r^2 \\
  \left( \vv{MM'} + t(\vv v' - \vv v) \right)^2 \leq r^2 \\
  {MM'}^2 + 2t(\vv v' - \vv v)\cdot \vv{MM'} + t^2 (\vv v' - \vv v)^2 \leq r^2 \\
  t^2 (\vv v' - \vv v)^2 + 2t(\vv v' - \vv v)\cdot \vv{MM'} + {MM'}^2 - r^2 \leq 0 \\
\end{gather*}
On retrouve un trinôme du second degré de coefficient positif.

Supposons le coefficient dominant nul, soit $\vv v' = \vv v$. L'inéquation se simplifie en:
\[
  MM' \leq r
\]
Physiquement, cela correspond au fait que les deux disques ont exactement le même mouvement,
et ne peuvent donc s'intersecter que s'ils sont déjà en cours d'intersection.

On suppose le coefficient dominant strictement positif. On note $t_{\text{min}}$
(resp. $t_{\text{max}}$) la plus petite (resp. la plus grande) des racines (éventuellement égales);
si elles existent.
Le temps d'intersection est la plus petite des ces valeurs possibles dans l'intervalle $[0,dt]$,
soit:
\begin{itemize}
  \item Aucune si le trinôme n'a pas de solution, ou si $t_\text{min} > dt$ ou $t_\text{max} < 0$
  \item $\max(0, t_\text{min})$ sinon.
\end{itemize}

\end{document}
